\documentclass{article}
    \usepackage[utf8]{inputenc}
    \usepackage{polski}
    \usepackage[polish]{babel}
    \usepackage{bbm}
    \usepackage{graphicx}    
    \usepackage{caption}
    \usepackage{subcaption}
    \usepackage{epstopdf}
    \usepackage{amsmath}
    \usepackage{amsthm}
    \usepackage{hyperref}
    \usepackage{url}
    \usepackage{comment}
    \newtheorem{defi}{Definicja}
    \newtheorem{twr}{Twierdzenie}
    \usepackage{listings}
    \usepackage{float}
    
    
    \author{Michał Martusewicz 282023}
    \date{Wrocław, \today}
    \title{\textbf{"Lata i strzela"}  \\ Dokumentacja}
    
    \begin{document}
    \maketitle
    \section{Strona użytkowa}
    Projekt to prosta gra zręcznościowa. Zadaniem gracza jest zestrzelenie jak największej liczby przeciwników. Gracz porusza statkiem, będącym czerwoną kropką. W grze są pewne utrudnienia: jest grawitacja (czyli przeciwnicy spadają coraz szybciej) oraz kolory. Przeciwnika danego koloru można zestrzelić wyłącznie strzałem o tym samym kolorze. Gra kończy się, gdy któryś z przeciwników dotknie dolnej części planszy. Przeciwnicy są różnej wielkości. Punkty przyznawane są następująco: 1 punkt za każde trafienie przeciwnika i 5 za ostateczne jego zniszczenie. Po każdym trafieniu przeciwnik się zmniejsza, po ostatnim znika zupełnie.
    Kolejnym utrudnieniem jest ograniczenie ilości amunicja jaka może być w jednym czasie na planszy.
    Dodatkowo z pewnym prawdopodobieństwem pojawiają się bonusy:
    \begin{itemize}
    \item tripple shot - potrójny strzał - przez pewien czas użytkownik strzela w trzech kierunkach jednocześnie;
    \item infinity ammo - nieskończona amunicja - przez pewien czas gracz może wystrzelić dowolnie wiele amunicji;
    \item red shot - czerwony strzał - strzał tego kolory może niszczyć przeciwników wszystkich kolorów;
    \item reversed grav - odwrócona grawitacja - przeciwnicy przestają spadać;
    \item less enemys - mniej przeciwników - pojawia się dwa razy mniej przeciwników.
    \end{itemize}
    Aby zdobyć bonus, należy go dotknąć swoim statkiem.
    \section{Strona techniczna}
    Projekt zawiera klasę dot, po której dziedziczą wszystkie klasy wszystkich obiektów: bonus, enemy, player i shot. Głównymi metodami tych klas to new, update i draw, które definiują tworzenie obiektów, ich aktualizację i rysowanie. Do tego jest klasa box, która odpowiada za stworzenie ścian dookoła planszy.
    \section{Uruchamianie + wykorzystanie}
    Aby uruchomić grę, należy wpisać polecenie 
    \begin{lstlisting}
    love path
    \end{lstlisting}
    gdzie path to ścieżka do folderu gry.
    Po uruchomieniu polecenia ukaże się okienko z grą. Aby rozpocząć grę należy nacisnąć dowolny klawisz alfanumeryczny. W czasie gry można wcisnąć klawisz "h" - ukaże się tekst pomocy. Klawisz "Esc" zamyka okno z grą. 
    Sterowanie w czasie gry: 
    \begin{itemize}
    \item Strzałki - poruszanie statkiem
    \item z, x, c, v - zmiana koloru strzału na odpowiednio: pomarańczowy, zielony, niebieski, fioletowy.
    \item spacja - strzał
    \end{itemize}
    
    
    \end{document}